\chapterimage{./Images/head3.jpg} % Chapter heading image
\chapter{Deterministic \& Randomized Algorithms}

\section{Asymptotics Contd.}
\begin{definition}
    [Monte Carlo Algorithms]
    A Monte Carlo algorithm is an algorithm that may return an incorrect answer with a small probability. The algorithm is said to be a Monte Carlo algorithm if it always runs in a fixed amount of time and the probability of returning an incorrect answer is at most $1/2$.
\end{definition}

\begin{definition}
    [Las Vegas Algorithms]
    A Las Vegas algorithm is an algorithm that always returns the correct answer. The algorithm is said to be a Las Vegas algorithm if it always runs in a fixed amount of time and the expected running time is finite.
\end{definition}

\begin{remark}
    In algorithms, we optimize time and memory usage. Memory is relatively cheap nowadays, so we can afford to use more memory. However, time is a very important factor. We can use randomness to reduce the time complexity of an algorithm.
\end{remark}
\subsection*{Time Complexity}
\begin{itemize}
    \item Best Case: The minimum time taken by the algorithm for an input size of $n$.
    \item Worst Case: The maximum time taken by the algorithm for an input size of $n$.
    \item Average/Expected Case: The average time taken by the algorithm for an input size of $n$.
    \item Amortized: The average time taken by the algorithm for a sequence of operations.
\end{itemize}

\section{Deterministic QuickSort}
\begin{definition}
    [QuickSort]
    QuickSort is a divide-and-conquer algorithm. It works as follows:
    \begin{enumerate}
        \item Choose a pivot element $p$ from the array.
        \item Partition the array into three parts: $L$ (elements less than $p$), $E$ (elements equal to $p$), and $G$ (elements greater than $p$).
        \item Recursively sort $L$ and $G$.
        \item Concatenate $L$, $E$, and $G$ to get the sorted array.
    \end{enumerate}
\end{definition}
% \begin{algorithm}[H]
%     \SetAlgoLined
%     \KwIn{Array $A$ of size $n$}
%     \KwOut{Sorted array $A$}
%     \If{$n \leq 1$}{
%         \Return $A$\;
%     }
%     $p \gets A[0]$\;
%     $L \gets \{x \in A \mid x < p\}$\;
%     $E \gets \{x \in A \mid x = p\}$\;
%     $G \gets \{x \in A \mid x > p\}$\;
%     \Return $\text{QuickSort}(L) + E + \text{QuickSort}(G)$\;
%     \caption{QuickSort}
% \end{algorithm}

\section{Logarithms}
\begin{definition}
    [Logarithm]
    \[
        \log_b a = c \iff b^c = a
    \]
    where $a, b, c \in \mathbb{R}$ and $b > 0, b \neq 1$.
\end{definition}
\begin{definition}
    [Properties of Logarithms]
    \begin{align*}
        \log_b 1                          & = 0                               \\
        \log_b b                          & = 1                               \\
        \log_b (a \cdot c)                & = \log_b a + \log_b c             \\
        \log_b \left( \frac{a}{c} \right) & = \log_b a - \log_b c             \\
        \log_b a^c                        & = c \cdot \log_b a                \\
        \log_b a                          & = \frac{1}{\log_a b}              \\
        \log_b(\frac{1}{a})               & = -\log_b a                       \\
        a^{\log_b c}                      & = c^{\log_b a}                    \\
        \log^(5)n                         & = \log(\log(\log(\log(\log(n)))))
    \end{align*}
\end{definition}

\begin{definition}
    [Iterated Logarithm]
    \[
        \log^(i) = \begin{cases}
            n                  & \text{if } i = 0 \\
            \log(\log^{(i-1)}) & \text{if } i > 0
        \end{cases}
    \]

\end{definition}

\begin{definition}
    \[\log^{*} n = \min \{ i \geq 0 \mid \log^{(i)} n \leq 1 \}\]
    A very slow-growing function.
\end{definition}

\begin{example}
    \[
        \log^{*} 2 = 1
    \]
    \[
        \log^{*} 2^2 = 1 + \log^{*} 2 = 2
    \]
    \[
        \log^{*} 2^4 = 3
    \]
\end{example}

\begin{definition}
    [Fibonacci Numbers]
    \[
        F_n = F_{n-1} + F_{n-2}
    \]
    \[
        F_0 = 0, F_1 = 1
    \]
    In terms of the golden ratio:
    \[
        F_n = \frac{\phi^n - \hat{\phi}^n}{\sqrt{5}}
    \]
\end{definition}

\begin{definition}
    [Golden Ratio]
    \[
        \phi = \frac{1 + \sqrt{5}}{2}
    \]
    \[
        \hat{\phi} = \frac{1 - \sqrt{5}}{2}
    \]
\end{definition}

\section{Summations}
\begin{definition}
    [Arithmetic Series]
    \[
        \sum_{i=1}^{n} i = \frac{n(n+1)}{2}
    \]
    \[
        \sum_{i=1}^{n} i^2 = \frac{n(n+1)(2n+1)}{6}
    \]
    \[
        \sum_{i=1}^{n} i^3 = \left( \frac{n(n+1)}{2} \right)^2
    \]
\end{definition}

\begin{definition}
    [Geometric Series]
    \[
        \sum_{i=0}^{n} a^i = \frac{a^{n+1} - 1}{a - 1}
    \]
    \[
        \sum_{i=0}^{\infty} a^i = \frac{1}{1 - a}
    \]
\end{definition}

\begin{definition}
    [Infinite Series]
    \[
        \sum_{i=0}^{\infty} x^k = \frac{1}{1 - x} \text{ when} |x| < 1
    \]
\end{definition}

\begin{example}
    [Summation]
    Show $ \sum_{k=0}^\infty kx^k = \frac{x}{(1-x)^2} $. \\
    You can differentiate both sides to prove this.
    \begin{align*}
        \sum x^k      & = \frac{1}{1-x} \quad \text{differentiate}     \\
        \sum kx^{k-1} & = \frac{1}{(1-x)^2} \quad \text{multiply by x} \\
        \sum kx^k     & = \frac{x}{(1-x)^2}
    \end{align*}

\end{example}

\subsection*{Telescoping Summation}
\begin{definition}
    [Telescoping Summation]
    \[
        \sum_{i=1}^{n} (a_i - a_{i-1}) = a_n - a_0 \quad \text{or} \quad \sum_{i=1}^{n} (a_i - a_{i+1}) = a_0 - a_n
    \]
\end{definition}

\begin{example}
    [Telescoping Summation]
    \[
        \sum_{k=1}^{n-1} \frac{1}{k+1} = \sum \frac{1}{k} - \frac{1}{k+1} = 1 - \frac{1}{n}
    \]
\end{example}

\begin{definition}
    [Bionomial Theorem]
    \[
        (x + y)^n = \sum_{k=0}^{n} \binom{n}{k} x^{n-k} y^k
    \]
    \[
        \binom{n}{k} = \frac{n!}{k!(n-k)!}
    \]
\end{definition}