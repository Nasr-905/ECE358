\chapterimage{./Images/head2.jpg} % Chapter heading image
\chapter{The Introduction}
\section{Administrative Information}

\begin{itemize}
    \item Similar to ECE345, may be able to substitute lectures
    \item 5 homeworks, no extensions
    \item Textbook: CLRS (Cormen, Leiserson, Rivest, Stein)
    \item Course email: ece.algorithms+358@gmail.com
    \item Homework: 20\%
    \item \begin{itemize}
        \item Groups of 2-3
        \item You can switch Groups
    \end{itemize}
    \item Midterm: 30\%
    \item \begin{itemize}
        \item Open Textbook + Notes
    \end{itemize}
    \item Final: 45\%
    \item \begin{itemize}
        \item Open Textbook + Notes
    \end{itemize}
\end{itemize}

\section{Course Overview}
\begin{itemize}
    \item Background (Discrete Math, ~5 lectures): Asymptotic Analysis, recurrences, summations, graph/trees, permutation \& combinations
    \item Sorting (~4 lectures): Quick, Heap, Radix/Counting, Lower Bound
    \item Binary Search Trees (~3 lectures): Red-Black, searching min/max
    \item Hashing (~ 1 lecture)
    \item Greedy Algorithms (~2 lectures)
    \item Dynamic Programming (~2 lectures)
    \item \textbf{Midterm}
    \item Amoratized analysis \& splay trees (~3 lectures)
    \item Grph Algorithms (~3 lectures)
    \item \begin{itemize}
        \item Basic (~1 lecture)
        \item MSTs (~1 lecture)
        \item Shortest Paths (~1 lecture)
        \item Max Flow (~2 lecture)
    \end{itemize}
    \item History of Computations (~1 lecture)
    \item NP-Completeness (~5 lectures)
    \item Blockchain \& Cryptography (After hours)
\end{itemize}

\section{Asymptotics}
\begin{definition}
    {Big-O Notation, Upper Bound}
    We say that $f(n) = O(g(n))$ \textit{iff} $O(g(n)) = f(n): \exists$ positive constants $c$ and $n_0$ such that $0 \leq f(n) \leq c\times g(n)$ for all $n \geq n_0$
\end{definition}
% Functions c times g(n) upper bound, f(n), and h(n) lower bound are plotted using tikz 
\begin{figure}[H]
    \centering
    \begin{tikzpicture}
        \draw[->] (0,0) -- (10,0) node[right] {$n$};
        \draw[->] (0,0) -- (0,10) node[above] {$f(n)$};
        \draw[domain=0:10,smooth,variable=\x,blue] plot ({\x},{\x}) node[right] {$f(n)$};
        \draw[domain=0:10,smooth,variable=\x,red] plot ({\x},{0.5*\x}) node[right] {$g(n)$};
        \draw[domain=0:10,smooth,variable=\x,green] plot ({\x},{0.25*\x}) node[right] {$h(n)$};
    \end{tikzpicture}
\end{figure}
% add a dotted n_0 line near the origin (n_0) is small

Essentialy, we don't care about small value, we care about the upper bound of the functions $f(n)$ and $g(n)$ as $n$ approaches infinity.

\begin{example}
    \begin{align*}
        f(n) &= 3n + 7 \in O(n)\\
        13n + 7 &\leq 14n, n_0 = 7, c = 14
    \end{align*}
    We can see that $f(n)$ is bounded by $O(n)$ because $f(n) \leq c\times g(n)$ for all $n \geq n_0$. \\
    You can choose any $c$ and $n_0$ that satisfies the inequality.
\end{example}

\begin{example}
    \begin{align*}
        n! = 1\times 2 \times 3 \ldots n \leq n \times n \cdots n = O(n^n)\\
    \end{align*}
\end{example}

\begin{definition}
    {Big-$\Omega$ Notation, Lower Bound}
    We say that $f(n) = \Omega(h(n))$ \textit{iff} $O(h(n)) = f(n): \exists$ positive constants $c$ and $n_0$ such that $0 \leq c\times h(n) \leq f(n)$ for all $n \geq n_0$
\end{definition}

\begin{example}
    [Arithmetic Sequence]
    \begin{align*}
        1 + 2 + 3 + \cdots + n \geq (\frac{n}{2}) + (\frac{n}{2} + 1) + \cdots + (n) \geq \frac{n}{2}\times \frac{n}{2} = (c = \frac{1}{4}) - n^2 = \Omega(n^2), n_0 \geq 1
    \end{align*}
\end{example}

\begin{definition}
    {Big-$\Theta$ Notation, Tight Bound}
    We say that $f(n) = \Theta(g(n))$ \textit{iff} $O(g(n)) = f(n) \land \Omega(g(n)) = f(n)$
\end{definition}

\begin{example}
    \begin{align*}
        1 + 2 + 3 + \cdots + n = \frac{n(n+1)}{2} = 
        \frac{n^2}{2} + \frac{n}{2} + 1= \Theta(n^2)
    \end{align*}
\end{example}

\begin{example}
    [Summation]
    Show that $\sum_{i=1}^{n} i^k = \Theta(n^{k+1})$
    \begin{align*}
        \sum_{i=1}^{n} i^k & \leq \sum_{i=1}^{n} n^{k} = n^{k+1} = O(n^{k+1})\\
    \end{align*}
    Therefore, $\sum_{i=1}^{n} i^k = O(n^{k+1})$, now we need to show that $\sum_{i=1}^{n} i^k = \Omega(n^{k+1})$
    \begin{align*}
        2f(n) =& \sum_{i=1}^{n}i^k + \sum_{i=1}^{n}(n-i+1)^k  \\
        = & 1 + 2^k + 3^k + \cdots + n^k + n^k + (n-1)^k + \cdots + 1^k\\
    \end{align*}
    \begin{align*}
        \sum_{i=1}^{n} i^k & \geq \sum_{i=\frac{n}{2}}^{n} i^k \geq \sum_{i=\frac{n}{2}}^{n} (\frac{n}{2})^k = \frac{n}{2} \times (\frac{n}{2})^k = \frac{n^{k+1}}{2^{k+1}} = \Omega(n^{k+1})
    \end{align*}
\end{example}

\subsection*{Cookbook}
\begin{definition}
    [Transitivity]
    If $f(n) = \Theta(g(n))$ and $g(n) = \Theta(h(n))$, then $f(n) = \Theta(h(n))$
\end{definition}


\begin{definition}
    [Symmetry]
    \begin{align*}
        f(n) &= \Theta(g(n)) \text{ \textit{iff} } g(n) = \Theta(f(n)) \\
        f(n) &= O(g(n)) \text{ \textit{iff} } g(n) = \Omega(f(n))
    \end{align*}
\end{definition}

\begin{theorem}
    [Other Properties]
    \begin{align*}
        n^a & \in O(n^b) \text{ \textit{iff} } a \leq b\\
        \log_a(n) & \in O(\log_b(n)) \forall a, b \\
        c^n & \in O(d^n) \text{\textit{iff}} c \leq d \\
        \text{If } f(n) & \in O(g(n)) \text{ and } h(n) \in O(k(n)) \text{ then } f(n) + h(n) \in O(g(n) + k(n))\\
    \end{align*}
    
\end{theorem}